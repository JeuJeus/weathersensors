%!TEX root = ../Thesis.tex
\section{Schlussbetrachtung}\label{Schlussbetrachtung}

\subsubsection{Bewertung}
Alle Muss-Features wurden erfolgreich implementiert. Auch die meisten Kann-Features wurden umgesetzt. Lediglich die
Wetterprognose wurde nicht entwickelt, da sie nach weiterer Recherche aus den gewonnenen Daten zu ungenau gewesen wäre.
Alle Features wurden gemäß der Projektplanung fristgerecht fertiggestellt. Sämtliche Deadlines konnten insbesonders
durch den frühen Projektstart eingehalten werden. Dies bezieht sich auf einzelne Teilaufgaben sowie das Gesamtprojekt.
\\
Die letzten Releases der Anwendung liefen über einen Zeitraum von über vier Wochen stabil auf dem Produktivsystem mit
einer Uptime von 100\%. Darüber hinaus gab es keine Fehler in den automatischen sowie manuellen Tests des Front- und
Backends.
\\
Die Qualität der erstellten Anwendung wird von den Entwicklern selbst folgendermaßen eingeschätzt:
\begin{longtable}{|p{0.14\textwidth}|p{0.15\textwidth}|p{0.09\textwidth}|p{0.15\textwidth}|p{0.17\textwidth}|p{0.15\textwidth}|}
	\caption{Bewertung der Anwendung}\\
	\hline
	Änderbarkeit & Benutzbarkeit & Effizienz & Funktionalität & Übertragbarkeit & Zuverlässigkeit\\
	\hline
	7 & 8 & 8 & 8 & 9 & 9 \\
	\hline
\end{longtable}
Die Bewertung richtet sich nach ISO/IEC 9126. Es folgt eine kurze Erläuterung der einzelnen Bewertungen.
Der am schlechtesten bewertete Punkt ist die Änderbarkeit. Diese ist auf die verwendete JavaScript Programmiersprache
zurückzuführen. Diese ist aufgrund ihrer schwachen Typisierung und inkonsistenten Implementierung anfällig für
Unübersichtlichkeit bei wachsenden Anwendungen. Sie wird jedoch für mehr als ausreichend für das Projekt in der jetzigen
Größe angesehen. Im Gegensatz dazu bietet sie zusammen mit Node.js, dem standardmäßigen Non-Blocking-IO sowie
Asynchronität eine sehr gute Effizienz für parallele Aufgaben, wie das -in diesem Fall- bereitstellen eines Webservers.
Die Fehleranfälligkeit von JavaScript wurde durch automatisierte Tests möglichst ausgeglichen, wodurch die hohe Zuverlässigkeit
entstanden ist. Die Übertragbarkeit wird als sehr hoch eingeschätzt, da die Anwendung als Docker Image auf sämtlichen modernen
Serverumgebungen problemlos installiert werden kann.
Die Funktionalität und Benutzbarkeit entstammen unserer subjektiven Einschätzung durch manuelle Tests sowie der von
dritten Testpersonen.



\subsubsection{Fazit}