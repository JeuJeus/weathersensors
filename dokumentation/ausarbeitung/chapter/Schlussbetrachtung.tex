%!TEX root = ../Thesis.tex
\section{Schlussbetrachtung}\label{Schlussbetrachtung}

\subsection{Bewertung}\label{subsec:bewertung}
Wie in \autoref{anh:sollist} zu erkennen ist, wurden alle Muss- sowie Kann-Features erfolgreich implementiert.
Lediglich die Wetterprognose wurde nicht entwickelt, da sie nach weiterer Recherche aus den gewonnenen Daten zu ungenau gewesen wäre.
Eine grobe Trendanalyse der letzten Datenpunkte wird jedoch in \autoref{subsec:ausblick} erläutert.
Alle Features wurden gemäß der Projektplanung fristgerecht fertiggestellt.
Sämtliche Deadlines konnten insbesonders durch den frühen Projektstart eingehalten werden.
Dies bezieht sich auf einzelne Teilaufgaben sowie das Gesamtprojekt.

Durch erste Erfahrungen der Entwickler mit JavaScript verlief der Einstieg in eine produktive Entwicklung zügig.
\\
Leider wurde das Team durch die exklusiv digitale Kommunikation - bedingt durch die anhaltende Corona-Pandemie - in der
Zusammenarbeit etwas gebremst.
Auch die - sonst übliche - hervorragende Arbeitsatmosphäre wurde dadurch verschlechtert.
\\
Die letzten Releases der Anwendung liefen über einen Zeitraum von über vier Wochen stabil auf dem Produktivsystem mit
einer Uptime von 100\%.
Darüber hinaus gab es keine Fehler in den automatischen sowie manuellen Tests des Front- und Backends.
\\
Die Qualität der erstellten Anwendung wird von den Entwicklern selbst folgendermaßen eingeschätzt (1-10 Punkte):
\begin{longtable}{|p{0.14\textwidth}|p{0.15\textwidth}|p{0.09\textwidth}|p{0.15\textwidth}|p{0.17\textwidth}|p{0.15\textwidth}|}
    \caption{Bewertung der Anwendung} \\
    \hline
    Änderbarkeit & Benutzbarkeit & Effizienz & Funktionalität & Übertragbarkeit & Zuverlässigkeit \\
    \hline
    7            & 8             & 9         & 8              & 9               & 9               \\
    \hline
\end{longtable}
Die Bewertung richtet sich nach ISO/IEC 9126\footnote{\cite{iso9126}}.
Es folgt eine kurze Erläuterung der einzelnen Bewertungen.
Das am schlechtesten bewertete Kriterium ist die Änderbarkeit.
Diese ist auf die verwendete JavaScript Programmiersprache zurückzuführen.
Diese ist aufgrund ihrer schwachen Typisierung und inkonsistenten Implementierung anfällig für Unübersichtlichkeit bei wachsenden Anwendungen.
Sie wird jedoch für mehr als ausreichend für das Projekt in der aktuellen Größe angesehen.
Im Gegensatz dazu bietet sie zusammen mit Node.js, dem standardmäßigen Non-Blocking-IO sowie
Asynchronität eine sehr gute Effizienz für parallele Aufgaben, wie das - in diesem Fall - bereitstellen eines Webservers.
Die Fehleranfälligkeit von JavaScript wurde durch automatisierte Tests möglichst ausgeglichen, wodurch die hohe Zuverlässigkeit
entstanden ist.
Die Übertragbarkeit wird als sehr hoch eingeschätzt, da die Anwendung als Docker Image auf sämtlichen modernen
Serverumgebungen problemlos installiert werden kann.
Die Funktionalität und Benutzbarkeit entstammen unserer subjektiven Einschätzung durch manuelle Tests sowie der von
dritten Testpersonen.
\\

\subsection{Ausblick}\label{subsec:ausblick}
Über die von uns implementierte Funktionalität hinaus, können wir uns die Ergänzung um eine Trendanzeige vorstellen.
Diese war im Rahmen unseres Projektes zeitlich nicht umsetzbar, könnte aber im Anschluss ergänzend hinzugefügt werden.
Sinn dieser wäre es, anhand von Regression\footnote{Vgl. \cite{regression}} eine \enquote{Wetterprognose} aufzustellen.
Hierzu sollte für die jeweiligen Sensorwerte ein Trend in der Entwicklung berechnet werden.
Das liesse sich anhand einer Regressionsgeraden der \enquote{n}-letzten\footnote{Dieser Wert ist zum einen abhängig vom Sendeintervall, zum anderen sollte dieser im Zuge der Implementierung evaluiert werden, um eine sinnvolle Prognose zu erhalten.} Datenpunkte umsetzen (Siehe Anhang \ref{prognose}).
Hierdurch würde unsere Anwendung um sinnvolle Funktionalität erweitert welche dem Nutzer durch klar erkennbare Prognosen einen Mehrwert bieten würde.

\subsection{Fazit}\label{subsec:fazit}
Im Soll-ist Vergleich in \autoref{anh:sollist} ist zu sehen, dass die Produktivität innerhalb des Projekts hervorragend war. Das Team bietet somit eine
gute Grundlage für weitere Zusammenarbeiten. Es wird gehofft, dass zukünftig wieder eine persönliches Zusammentreffen möglich ist,
um auch mal auf den gemeinsamen Erfolg einen Bölkstoff zischen zu können.


