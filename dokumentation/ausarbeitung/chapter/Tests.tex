%!TEX root = ../Thesis.tex
\section{Tests}\label{Tests}
\subsection{Automatische Tests}
Für viele Funktionen der Anwendung wurden Unit-Tests geschrieben.
Im Backend sowie Frontend wurden die Frameworks Chai und Mocha verwendet.
Chai ist eine Assertion Library, während Mocha ein komplettes Test-Framework ist.
Die Struktur der Tests wird somit durch Mocha vorgebenen.
Durch Chai werden lediglich die Assertions evaluiert.
\\
Die Testklassen heißen jeweils testMain.js und befinden sich in den src/test Ordnern des Front- und Backends.

\subsection{Manuelle Klicktests}
Es wurden manuelle Tests der Benutzoberfläche ausgeführt, um etwaige Fehler dort zu entdecken.
Diese wurden dokumentiert.
\subsubsection{Inhalte}
Es sollen folgende Inhalte dokumentiert werden:
\begin{itemize}
    \item Git Commit Hash (Programmversion)
    \item Git Branch
    \item Verwendetes Betriebssystem
    \item Verwendeter Browser inkl. Build
    \item Screenshots bei Darstellungsfehlern
    \item Bildschirmauflösung, insbesonders bei mobiler Ansicht
\end{itemize}
Darüber hinaus wurde die Anwendung in der mobilen (responsiven) Ansicht getestet, was ebenfalls dokumentiert wurde.


