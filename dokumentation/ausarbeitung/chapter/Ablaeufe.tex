%!TEX root = ../Thesis.tex


\section{Abläufe}\label{Ablaeufe}
In diesem Kapitel werden die Abläufe der jeweiligen Anwendungen (vereinfacht) dargestellt. Für die Komponenten
Backend und NodeMCU wurde jeweils ein UML-Diagramm erstellt. Für das Frontend wurde darauf verzichtet, da bereits
ein Use-Case-Diagramm besteht und der technische Aufbau sehr überschaubar ist. Die UML-Diagramme befinden sich in Anhang
\ref{anh:ablaeufe}.

\subsection{Backend}
In \autoref{fig:backend_activity} ist ein Aktivitätendiagramm zu sehen. In diesem ist der simple Aufbau des Backends
zu erkennen. Hierbei handelt es sich um einen klassischen Webserver, der Requests empfängt und verarbeitet.

\subsection{NodeMCU}
Das Zustandsdiagramm ist in \autoref{fig:zust_diag_nodemcu_setup}, \autoref{fig:zust_diag_nodemcu_loop} und
\autoref{fig:zust_diag_nodemcu_send} zu sehen. Es wurde zu Gunsten der Übersichtlichkeit in drei Teile aufgeteilt.
Nach dem Boot wird in die setup-Funktion gesprungen.
In dieser wird nach einem BME280 Sensor gesucht.
Insofern einer gefunden wurde, wird die WLAN-Verbindung aufgebaut.
Anschließend wird die loop-Funktion endlos in einer Schleife aufgerufen.
Diese beginnt mit dem Auslesen der Sensordaten, welche in eine Liste gespeichert werden.
Wenn die Liste voll ist, wird das erste Element zuvor entfernt.
Nach Prüfung, ob die WLAN-Verbindung vorhanden ist, wird die sendCachedData-Funktion aufgerufen.
In dieser wird über alle Listenelemente iteriert.
In jeder Iteration wird eine HTTP-Verbindung zum Backend aufgebaut.
Anschließend wird ein eventuell inkorrekter Timestamp (mit Wert = 0) korrigiert.
Danach wird der JSON-String für den HTTP-Body aus den Daten erstellt.
Nach anschließendem Versenden wird der HTTP-Code ausgewertet.
Bei einem erwarteten Wert wird das Element aus der Liste entfernt.
Zuletzt wird die HTTP-Verbindung geschlossen und in die nächste Iteration gegangen.
Mit dem Ende der Iterationen endet die Funktion, womit in die loop-Funktion zurückgekehrt wird.
Nach Warten einer Verzögerung wird diese erneut (endlos) aufgerufen.
