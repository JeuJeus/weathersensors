%!TEX root = ../Thesis.tex


\section{Installation}\label{Installation}

\subsection{ESP8266}\label{ESP}
Zur Erfassung der Wetterdaten wird ein BME280 Sensor verwendet, welcher von einem Board mit ESP8266 Mikrocontroller und NodeMCU 1.0 Firmware angesteuert wird.
Auf diesem Mikrocontroller kann der Sourcecode \textit{nodemcu.ino} ausgeführt werden.
Zum Kompilieren sollte die Arduino IDE verwendet werden, in der zuvor die Treiber für den ESP8266 installiert werden müssen.
Dazu muss in den Voreinstellungen folgende Boardverwalter-URL hinzugefügt werden: \textit{\href{http://arduino.esp8266.com/stable/package\_esp8266com\_index.json}{Boardverwalter-URL}}.
Darüber hinaus müssen folgende Bibliotheken über die integrierte Bibliotheksverwaltung installiert werden:

\begin{table}[hbt]
    \centering
    \begin{minipage}[t]{.5\textwidth}
        \caption{Verwendete Arduino-Bibliotheken}
        \begin{tabular}{|l|c|}
            \hline
            \textbf{Bibliothek}                   & \textbf{Version} \\
            \hline
            WiFi (by Arduino)                     & 1.2.7            \\
            \hline
            Adafruit BME280 Library (by Adafruit) & 2.1.1            \\
            \hline
            Adafruit Unified Sensor (by Adafruit) & 1.1.4            \\
            \hline
            EasyNTPClient (by Harsha Alva)        & 1.1.0            \\
            \hline
            LinkedList (by Ivan Seidel)           & 1.2.3            \\
			\hline
		\end{tabular}
	\source{Eigene Darstellung}
\label{tab:usedArduinoLibs}
\end{minipage}
\end{table}

Bevor der Quellcode kompiliert wird, müssen die folgenden Konstanten auf die lokalen Gegebenheiten angepasst werden:

\begin{itemize}
	\item \textit{SERVER\_TO\_CONNECT}
	\item \textit{SSID}
	\item \textit{WIFI\_PASSWORD}
\end{itemize}

Der BME280 Sensor und der ESP8266 müssen folgendermaßen verbunden werden:

\begin{table}[hbt]
	\centering
	\begin{minipage}[t]{.5\textwidth}
		\caption{Pin Layout für Verbindung ESP mit BME}
	\begin{tabular}{|l|l|}
		\hline
		\textbf{ESP8266 Pin}	& \textbf{BME280}  \\
		\hline
		3.3V & VIN \\
		\hline
		G & GND \\
		\hline
		D1 & SCL \\
		\hline
		D2 & SDA \\
		\hline
	\end{tabular}
	\\\source{Eigene Darstellung}
\label{tab:espBmePinout}
\end{minipage}
\end{table}

Die exakten Geräte sind:

\begin{itemize}
    \item AZDelivery NodeMCU Lua Lolin V3 Module ESP8266 ESP-12F WIFI
    \item AZDelivery GY-BME280
\end{itemize}

\subsection{Backend und Frontend}

Backend sowie Frontend werden mithilfe von Docker deployed.
Im Frontend ist hierzu die Backend URL in der Klasse \textit{Constants.js} die Variable \textit{SERVER\_URI} anzupassen.
Die Images dafür lassen sich in den jeweiligen Modulen mit Hilfe der \textit{buildImageandTar.sh} Skripts erstellen.
Diese erstellen die Images und stellen sie in der lokalen Dockerumgebung zum Start zur Verfügung.
Darüber hinaus werden im Projekt-Root-Ordner tar-Bälle mit den jeweiligen Images hinterlegt.
Für diesen Schritt haben wir uns entscheiden um die Images einfach auf einem Server verfügbar zu machen ohne Docker Registries (z.B. Docker.io) in Anspruch nehmen zu müssen.

Der Standard Admin-Benutzername ist \enquote{Admin} und das Standard-Passwort \enquote{\$PASSWORD}. Das Passwort kann in \textit{router.js} geändert werden.

\textbf{Deployment Voraussetzungen unter Windows}

Das Projekt kann mithilfe der WSL 2 und Docker for Windows deployed werden.
Für die Vorbereitung muss zunächst eine WSL 2 eingerichtet werden.
Danach kann Docker for Windows mit den WSL 2-Komponenten installiert werden.
Nachdem Docker for Windows bereitgestellt wurde, kann die ausgewählte Linux Distribution in der WSL gestartet werden.
Danach sind in der WSL die Schritte für Linux auszuführen.

\textbf{Deployment Voraussetzungen unter Linux}

In Linux sind Docker sowie Node und npm durch den Distribution-spezifischen Package Manager zu installieren.

\textbf{Backend Start-Command:}

\textit{docker run -p 3000:3000 -v \$PATH\_TO\_DATABASE:/usr/src/app/db --name awe2-backend -it awe2/backend:abgabe}

\textit{\$PATH\_TO\_DATABASE} ist zu ersetzen mit dem Ordner, in welchem die Datenbank auf dem Host-System gespeichert werden soll.

\textbf{Frontend Start-Command:}

\textit{docker run -p 3344:3344 --name awe2-frontend -it awe2/frontend:abgabe}
