%!TEX root = ../Thesis.tex
\section*{Anhang}
\addcontentsline{toc}{section}{Anhang}
\fancyhead[R]{Anhang}

\anhangsverzeichnis

\anhang{Schnittstellen}\label{Anhang:Schnittstellen}

\subanhang{Antwort GET-Request auf \enquote{/weatherData}}
\begin{figure}[bht]
    \begin{lstlisting}[caption=Antwort GET-Request /weatherData, label=list:getWeatherData]
		{"sensors":[
			{"ID":1,
				"MAC_ADDRESS":"e0:98:06:86:23:bc",
				"LOCATION":"Julius (Paderborn)",
				"LAST_UPDATE":1605628770000
			},
			{"ID":2,
				"MAC_ADDRESS":"68:c6:3a:88:c0:cd",
				"LOCATION":"Jonathan (nun auch nach alledem in Paderborn)",
				"LAST_UPDATE":1605357215000
			},
			{"ID":3,
				"MAC_ADDRESS":"f4:cf:a2:d1:49:3e",
				"LOCATION":"Philipp (Paderborn)",
				"LAST_UPDATE":1605628927000
			}
			],
		"sensorData":[
			{"SENSOR_ID":1,
				"TIMESTAMP":1604251064000,
				"TEMPERATURE":21.610001,
				"AIRPRESSURE":996.586426,
				"HUMIDITY":60.304688
			},
			{"SENSOR_ID":1,
				"TIMESTAMP":1604251364000,
				"TEMPERATURE":21.610001,
				"AIRPRESSURE":996.608887,
				"HUMIDITY":60.868164
			}
			]
		}
    \end{lstlisting}
\end{figure}

\pagebreak

\subanhang{Antwort GET-Request auf \enquote{/sensorData/id/1}}
\begin{figure}[bht]
    \begin{lstlisting}[caption=Antwort GET-Request /sensorData/id/1, label=list:getSensorData1]
		{"sensorData":[
			{"SENSOR_ID":1,
				"TIMESTAMP":1604251064000,
				"TEMPERATURE":21.610001,
				"AIRPRESSURE":996.586426,
				"HUMIDITY":60.304688
			},
			{"SENSOR_ID":1,
				"TIMESTAMP":1604251364000,
				"TEMPERATURE":21.610001,
				"AIRPRESSURE":996.608887,
				"HUMIDITY":60.868164
			},
			{"SENSOR_ID":1,
				"TIMESTAMP":1604251666000,
				"TEMPERATURE":21.610001,
				"AIRPRESSURE":996.680603,
				"HUMIDITY":60.949219
			}
			]
		}
    \end{lstlisting}
\end{figure}

\pagebreak

\subanhang{Antwort GET-Request auf \enquote{/sensors}}
\begin{figure}[bht]
    \begin{lstlisting}[caption=Antwort GET-Request /sensors, label=list:getSensors]
	{"sensors":[
		{"ID":1,
			"MAC_ADDRESS":"e0:98:06:86:23:bc",
			"LOCATION":"Julius (Paderborn)",
			"LAST_UPDATE":1605629371000
		},
		{"ID":2,
			"MAC_ADDRESS":"68:c6:3a:88:c0:cd",
			"LOCATION":"Jonathan (nun auch nach alledem in Paderborn)",
			"LAST_UPDATE":1605357215000
		},
		{"ID":3,
			"MAC_ADDRESS":"f4:cf:a2:d1:49:3e",
			"LOCATION":"Philipp (Paderborn)",
			"LAST_UPDATE":1605629652000
		}
	]
}
    \end{lstlisting}
\end{figure}

\subanhang{Antwort GET-Request auf \enquote{/sensor/id/1}}
\begin{figure}[bht]
    \begin{lstlisting}[caption=Antwort GET-Request /sensor/id/1, label=list:getSensor]
	{
		"sensor": {
			"ID": 1,
			"MAC_ADDRESS": "e0:98:06:86:23:bc",
			"LOCATION": "Julius (Paderborn)"
		}
	}
    \end{lstlisting}
\end{figure}

\pagebreak

\anhang{Logdaten}\label{anhang:logdaten}
\textbf{Anmerkung:} Aus Datenschutzgründen sind alle IP-Adressen im Folgenden zensiert.
\subanhang{Beispielhafter Logauszug Backend}
\begin{figure}[bht]
    \begin{lstlisting}[caption=Beispielhafter Logauszug Backend, label=list:logBackend]
		18.11.2020, 09:41:09 - ERROR :
			POST REQUEST PARSING BODY FAILED FROM [$ZENSIERTE_IP_ADRESSE], REQUEST BODY: {
				"MACADDRESS":"68:c6:3a:88:c0:cd",
				"TIMESTAMP":"1605692160",
				"TEMPERATURE":-143.25,
				"AIRPRESSURE":1185.736206,
				"HUMIDITY":100
			}

		INSERT INTO SENSOR (MAC_ADDRESS, LOCATION)
			VALUES (?, "") EXCEPT
				SELECT MAC_ADDRESS, LOCATION FROM SENSOR WHERE MAC_ADDRESS = ?

		18.11.2020, 09:41:25 - INFO :
			GOT REQUEST TO [/weatherData] FROM [$ZENSIERTE_IP_ADRESSE]

		18.11.2020, 09:41:42 - INFO :
			GOT REQUEST TO [/sensorData/id/1?granularity=100] FROM [$ZENSIERTE_IP_ADRESSE]

		18.11.2020, 09:41:43 - INFO :
			GOT REQUEST TO [/sensors/] FROM [$ZENSIERTE_IP_ADRESSE]

		18.11.2020, 09:41:44 - INFO :
			GOT REQUEST TO [/sensor/id/1] FROM [$ZENSIERTE_IP_ADRESSE]
    \end{lstlisting}
\end{figure}

\subanhang{Beispielhafter Logauszug Frontend}
\begin{figure}[bht]
    \begin{lstlisting}[caption=Beispielhafter Logauszug Frontend, label=list:logFrontend]
		16.11.2020, 11:00:34 - INFO :
			FRONTEND STARTED

		16.11.2020, 11:08:58 - INFO :
			GOT REQUEST TO [/] FROM [$ZENSIERTE_IP_ADRESSE]

		16.11.2020, 11:09:54 - INFO :
			GOT REQUEST TO [/] FROM [$ZENSIERTE_IP_ADRESSE]

		16.11.2020, 11:09:54 - INFO :
			GOT REQUEST TO [/favicon.ico] FROM [$ZENSIERTE_IP_ADRESSE]

		16.11.2020, 12:50:18 - INFO :
			GOT REQUEST TO [/robots.txt] FROM [$ZENSIERTE_IP_ADRESSE]
    \end{lstlisting}
\end{figure}

\anhang{GUI}

\subanhang{Finales GUI Design}\label{gui-fertig}

Hier wird mit Blick auf den Prototypen sichtbar, wie geringfügig die Änderungen gegenüber diesem ausgefallen sind.

\begin{figure}[h!!]
    \centering
    \begin{minipage}[t]{1\textwidth}
        \caption{Finale GUI}
        \includegraphics[width=1\textwidth]{img/gui-fertig.png}\\
        \source{Eigene Darstellung}
        \label{fig:finale-gui}
    \end{minipage}
\end{figure}

\pagebreak

\subanhang{Datumsauswahl}\label{datumsauswahl}

Die Datums- sowie Zeitauswahl beruht auf der Library \enquote{Date Range Picker}\footnote{\cite{daterangepicker.2020}}.
Diese ermöglicht eine einfache, ohne Anleitung verständliche, Auswahl des anzuzeigenden Zeitbereiches.
Auch hier wurden die Farben durch unsere Pastell-Farb-Palette angepasst.

\begin{figure}[h!!]
    \centering
    \begin{minipage}[t]{1\textwidth}
        \caption{Datumsauswahl Popup (finale GUI)}
        \includegraphics[width=1\textwidth]{img/datumsauswahl.png}\\
        \source{Eigene Darstellung}
        \label{fig:datumsauswahl}
    \end{minipage}
\end{figure}

\pagebreak

\subanhang{Admininterface}\label{admininterface}

Das Admininterface wurde einfach gehalten und greift wie die Hauptseite auf \enquote{Font-Awesome}\footnote{\cite{fontawesome}} für die Icons zurück.
Alle zur Verfügung stehenden Sensoren werden hier innerhalb einer Tabelle aufgelistet, der Standort des Sensors lässt sich einfach durch ein Inputfeld mit zugehörigem Button anpassen.
Bei erfolgreicher Eingabe wird eine positive Rückmeldung eingeblendet.
Diese findet sich beispielhaft in Anhang \ref{popup}.

\begin{figure}[h!!]
    \centering
    \begin{minipage}[t]{1\textwidth}
        \caption{Admininterface (finale GUI)}
        \includegraphics[width=1\textwidth]{img/admin.png}\\
        \source{Eigene Darstellung}
        \label{fig:admin}
    \end{minipage}
\end{figure}

\pagebreak

\subanhang{Informationsmeldungen}\label{popup}

Diese Einblendung wird zur Information des Nutzers angezeigt.
Neben dieser Darstellung welche nach erfolgreicher Änderung des Sensorstandortes eingeblendet wird, existiert eine \enquote{Error}-Version.
Diese ist in Rot gehalten und zeigt Fehler auf.

\begin{figure}[h!!]
    \centering
    \begin{minipage}[t]{1\textwidth}
        \caption{Informationsmeldung\pagebreak
        (finale GUI)}
        \includegraphics[width=1\textwidth]{img/popup.png}\\
        \source{Eigene Darstellung}
        \label{fig:datumsauswahl}
    \end{minipage}
\end{figure}
